\subsection*{Дано}
\begin{enumerate}
    \item Ориентированный граф работ $G$ без циклов, в котором дуги - зависимости по данным, а вершины - задания. Вершин $n$, дуг $m$
    \item Вычислительная система, состоящая из $p$ различных процессоров
    \item Матрица $C_{ij}$ длительности выполнения работ на процессорах, $i=1 \dots n, j=1 \dots p$
    \item Матрица $D_{kl}$ передач данных между процессорами, $k=1 \dots p, l = 1 \dots p, D_{kk} = 0$
\end{enumerate}
\subsection*{Определение расписания}
Расписание программы определено, если
\begin{enumerate}
    \item Множества процессор и работ
    \item Привязка - всюду определенная на множестве работ функция, которая задает распределение работ по процессорам
    \item Порядок - заданные ограничения на последовательность выполнения работ и является отношением частичного порядка, удовлетворяющим условиям ацикличности и транзитивности. Отношение порядка на множестве работ, распределенных на один процессор, является отношением полного порядка.
\end{enumerate}

\subsection*{Требуется}
\begin{enumerate}
    \item Построить расписание $HP$, то есть для $i$-й работы определить время начала ее выполнения $s_i$ и процессор $p_i$ на которм она будет выполняться
    \item В расписании требуется минимизировать время выполнения набора работ, данных в графе $G$
    \item В задаче так же присутствуют дополнительные ограничения, котрым расписание обязано удовлетворять.
\end{enumerate}
\subsection*{Различные постановки задачи:}
\begin{enumerate}
    \item Задача с однородными процессорами (длительность выполнения работы не зависит от того, на каком процессоре она выполняется) и дополнительными ограничениями на количество передач:
          \begin{itemize}
              \item $CR = \frac{m_{ip}}{m} < 0.4$, где $m_{ip}$ - количество передач данных между работами на каждый процессор
              \item $CR2 = \frac{m_{2edg}}{m} < 0.05$, где $m_{2edg}$ - количество дуг, начальный и конечный узлы которых назначены на процессоры, не соединенных напрямую
          \end{itemize}
    \item Задача с однородными процессорами и дополнительным ограничением сбалансированности распределения работ:
          \begin{itemize}
              \item $BF = \lceil 100 \cdot \left( \frac{a_{max} \cdot p}{n} - 1 \right) \rceil < 10$, где $a_{max}$ - наибольшее, по всем процессорам, количество работ на процессоре
          \end{itemize}
    \item Задача с неоднородными процессорами, но без дополнительных ограничений на расписание
\end{enumerate}