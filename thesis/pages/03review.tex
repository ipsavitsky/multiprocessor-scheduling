Обзоры других алгоритмов приведены в \cite{Shakhbazyan_1981,Davis_2011}
\subsection{Жадные алгоритмы}
Жадные алгоритмы подразумевают декомпозицию задачи на ряд более простых подзадач. На каждом шаге решение принимается исходя из принципа получения оптимального решения для очередной подзадачи. То есть, на каждом шаге алгоритм делает выбор, оптимальный с точки зрения получения решения очередной подзадачи, предполагая, что эти локально-оптимальные решения приведут к приемлемому решению задачи. Какие-либо жадные стратегии, гарантированно получающие оптимальное расписание, на настоящий момент времени неизвестны, за исключением небольшого числа вариантов задач составления расписаний не принадлежащих к классу NP-полных. Например, известен жадный алгоритм, получающий точное решение для задачи обслуживания одним процессором максимального числа работ из заданного набора работ с фиксированными сроками начала и окончания \cite{Cormen}. Набор локальных критериев оптимизации сильно зависит от класса архитектуры. Для архитектур, в которых возможно последействие (распределяемый в расписание рабочий интервал оказывает влияние на времена инициализации ранее распределенных рабочих интервалов) возникает проблема выбора локальных критериев оптимизации, позволяющих учесть эффект последействия (на настоящий момент времени какие-либо обоснованные решения этой проблемы не известны). Кроме того, единого локального критерия (или набора и способа их использования), приводящего к наилучшему конечному результату, для решения всех подзадач не существует. Более того, при усложнении архитектуры набор и способ использования локальных критериев оказывает более сильное влияние на конечный результат. Таким образом, применение жадных алгоритмов для составления расписаний классом архитектур без последействия или даже без разделяемых ресурсов, если их влияние на значение функции построения временной диаграммы не может быть локализовано, а также проблемой выбора критериев оптимизации индивидуально для каждой подзадачи.

\subsection{Жадные алгоритмы с процедурой ограниченного перебора}

Жадные алгоритмы в чистом  виде применяются редко из-за того, что в результате их работы часто получаются решения очень далекие от оптимальных, так как они склонны застревать в локальных минимумах оптимизирующей функции. Для того чтобы исправить этот недостаток был предложен подход \cite{Kostenko_2017}, сочетающий жадные алгоритм и ограниченный перебор. 

В основу предлагаемых в работе алгоритмов, сочетающих жадные стратегии и ограниченный перебор, положены следующие основные принципы:
\begin{itemize}
    \item на каждом шаге работы алгоритма делается локально-оптимальный выбор в соответствии с используемыми жадными критериями,
    \item на каждом шаге производится проверка того, что «жадный выбор не закрывает пути к оптимальному решению»,
    \item вызов процедуры ограниченного перебора, если проверка условия «жадный выбор не закрывает пути к оптимальному решению» дала отрицательный результат.
\end{itemize}
Предложенный метод построения алгоритмов сочетающих жадные стратегии и ограниченный перебор дает возможность задавать требуемый баланс между точностью и сложностью алгоритма путем выбора значения максимально допустимой глубины перебора. Метод допускает настройку на частную задачу путем подбора жадных критериев.

В разобранном в статье \cite{Kostenko_2017} алгоритме рассматривается классический вариант задачи построения многопроцессорного расписания без учета дополнительных критериев, однако эти критерии легко учитываются как дополнительные условия при проверке оптимальности текущего расписания, построенного жадным алгоритмом.

Этот алгоритм детерминирован и имеет несколько параметров, один из которых – глубина перебора в процедуре ограниченного перебора. При установке малой глубины перебора можно создать неточное, но верное начальное приближение расписания для других алгоритмов, например для генетических алгоритмов, имитации отжига или дифференциальной эволюции. Притом, алгоритм конструктивен, нет потребности задания критерия останова алгоритма.

В краевом случае, когда все задачи распределяются в соответствии с жадными критериями и процедура ограниченного перебора не вызывается алгоритм имеет сложность $O(nlogn)$.

Достоинства:
\begin{enumerate}
    \item Возможность балансировать между точностью и сложностью благодаря изменению глубины перебора.
    \item Не зависит от начальных условий, т.к. алгоритм детерминированный.
    \item Возможность получить расписание для поддерева работ.
\end{enumerate}


Недостатки:
\begin{enumerate}
    \item Необходимо подбирать жадные критерии в ходе эксперимента.
    \item Время работы алгоритма может быть большим, если процедура ограниченного перебора будет вызываться слишком часто.
\end{enumerate}
